\section{Product Perspective}
   
This product is supposed to be part of an open source, under the GNU general Public License. It is a CAD software for programmers i.e. you code to make models. Customizer is the idea given by the user's only to fullfill their needs and It had been in demand even before it consentment. The Customizer will extent this already powerfull system by allowing the user to design the genric models and modify the parameters without the need to modify the program itself. It will also provide a way to modify the set of parameters using cmd-line.
   
The following are the main features that are included in OpenSCAD'customizer

\begin{enumerate}
    \item \textbf{Cross platform support:} Offers operating support for most of the known and commercial operating systems in form of binaries and also it can be compiled on other platforms.
   
    \item \textbf{Allows to modify parameters using GUI:} The system allows the user to easily modify the parameters of the given scad program using the GUI instead of changing it
    manually.
   
    \item \textbf{Backward Compatibility:} OpenSCAD should be backward compatibility even after new features are implemented.
   
    \item \textbf{Easy to use syntax:} The syntax for creating the input widget, groups, and description in customizer. Should be easy to use.
   
    \item \textbf{Abiltiy to manage parameters:} It should be easy to manage the parameters by grouping them or giving them option to hide or unhide them.
   
    \item \textbf{Feature to save a different set of parameters:} Feature to save a different set of parameters with customized name should be provided. So, that given set could be used in future.
   
    \item \textbf{Feature to modify saved sets:} User should be able to modify,delete the already saved sets of parameters.
    
    \item \textbf{Feature to diable Customizer:} User should be able to diable this feature totally. So, that software should work like this feature doesn't exit.
   
\end{enumerate}  
